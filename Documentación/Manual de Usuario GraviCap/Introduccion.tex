\chapter{Introducción}
    \section{¿Qué es \textcolor{dark_violet}{GraviCap}?}
        \textcolor{dark_violet}{GraviCap} (Gravity Capacitor) es una batería de almacenamiento gravitatorio. Su finalidad es igual a una batería convencional, pero con un funcionamiento de generación de energía totalmente distinto e innovador.\par

    \section{Funcionamiento Principal}
        Esta batería está conformada por una estructura, la cual eleva un peso por medio de un motor, un sistema de poleas y un cable de acero, almacenando energía potencial gravitatoria en el peso. Cuando se necesita energía en el dispositivo que queramos cargar, el motor hará descender la masa de manera controlada, generando energía eléctrica gracias al movimiento del motor al ser movido por el cable.\par

    \section{Objetivos}
        Teniendo en cuenta como fenómeno mundial, la contaminación por baterías convencionales, ya sea de litio, de autos o de celulares, surge nuestro objetivo principal, darle al mundo una nueva alternativa de almacenamiento de energía por medio de nuestro producto, trayendo a nuestro país una nueva forma de batería, siendo ideal para usarla con energías renovables, no contaminantes y sustentables. Finalmente presentando al mundo una forma sustentable y no contaminante de almacenar energía.\par
        
    \section{Beneficios}
        \begin{itemize}
            \item Nuestra batería, al tener como factor principal aprovechar la energía potencial gravitatoria para la generación de energía eléctrica, es completamente sustentable y no contaminante para el medio ambiente.\par
            \item Esta forma de generación de energía es poco vista e implementada dentro del mundo y de las primeras utilizadas en Argentina. Siendo así, una nueva forma vanguardista de batería dentro de nuestro país.
            \item Puede aprovecharse lugares o infraestructuras en desuso para la instalación de la batería. Por ejemplo, minas abandonadas, contando con una gran profundidad para el recorrido del peso. También, puede aprovecharse una torre de control de un aeródromo o aeropuerto, contando con una gran altura para un eficiente funcionamiento y recorrido del peso. 
            \item El sistema de almacenamiento de energía sobrante de las fuentes utilizadas, garantiza para el usuario una estabilidad de energía en la batería en los momentos de pico o baja consumo aprovechando al máximo lo que genera.
            \item Gracias al sistema mecánico de \textcolor{dark_violet}{GraviCap} se producen menos desgastes en la estructura en la carga y descarga de energía, así proporcionando una vida útil más larga que una batería convencional y, por ende, menos coste de mantenimiento. 
            \item Permite monitorear el funcionamiento del sistema \textcolor{dark_violet}{GraviCap} por medio de una aplicación móvil.
        
        \end{itemize}